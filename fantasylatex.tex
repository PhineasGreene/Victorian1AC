\documentclass[11pt]{article}

\title{Ban Open Holding of Fracking Wastewater}
\author{Phineas Greene and LJ Oby}

\usepackage{soul}

\usepackage{fontspec}
\setmainfont{Cormorant Garamond}

\usepackage{lettrine} 
\renewcommand{\LettrineFontHook}{\fontspec{Cinzel Decorative}}
\renewcommand{\LettrineTextFont}{\fontspec{Cinzel Decorative}}

\usepackage[margin=1.2in]{geometry}
\addtolength{\topmargin}{-.5in}
\addtolength{\textheight}{2in}

\usepackage{titlesec}
\titleformat{\section}{\huge\bfseries}{}{-0.5em}{}
\titleformat{\subsection}{\large\bfseries}{}{0.5em}{}

\usepackage{pgfornament} 

\begin{document}
	\begin{center}
	\huge
	Ban Open Holding of Fracking Wastewater
	\end{center}
	\begin{center}
	\large
	Phineas Greene and LJ Oby
	\end{center}
	
	\begin{center}
	\pgfornament[width=16cm]{89}
	\end{center}
	
	\paragraph{}
	\begin{quotation}
	\lettrine{H}{ydraulic} \ul{fracturing, or fracking, is the process of injecting pressurized water, chemicals, and sand into the ground to extract shale oil and natural gas. While fracking has transformed energy production in the U.S., it requires a lot of water and thus creates a lot of wastewater, which must be treated and disposed of safely.} In this video interview, Triplepoint’s resident expert, western regional manager Tom Daugherty, gives the lowdown on fracking wastewater and how frac ponds can be economically upgraded with MARS aeration. Read below for highlights and links to more information.''
	\textit{May 23, 2018 by Patrick Hill\footnote{http://www.triplepointwater.com/frac-ponds/ Patrick Hill is a writer for Triple Point}}
	\end{quotation}
	
	
	\paragraph{}
	Fracking has transformed the oil industry. As we see in this report from Patrick Hill on May 23, 2018, it is an amazing thing that comes with new responsibilities. As my partner and I will show you in today’s debate round, change is needed in the status quo. That is why my partner and I stand resolved that the United States Federal Government should substantially reform its energy policy. This brings me to our Definitions:

	\section{Definitions}
	
	\subsection{Definition 1: Fracking}
	\paragraph{}
	Fracking is defined as: ``the injection of fluid into shale beds at high pressure in order to free up petroleum resources (such as oil or natural gas)''\footnote{The Merriam-Webster.com Dictionary, Merriam-Webster Inc., https://www.merriam-webster.com/dictionary/fracking}

	\subsection{Definition 2: Fracking Pond}
	\paragraph{}
	Fracking ponds, also called Frack Ponds, Open Pits, Impoundments, or Open Holding Sites are defined as: ``A traditional fracking pond is basically a hole dug in the surface of the ground. Many companies will dig frac ponds on site near the well in order to easily dispose of frac fluid right there on the property.''\footnote{https://wwstanks.com/2016/10/10/what-is-a-fracking-pond-and-all-the-various-alternatives/ Well Water Solutions and Resources Inc. 2016/10/10}

	\subsection{Definition 3: The Resource Conservation and Recovery Act}
	\paragraph{}
	The Resource Conservation and Recovery Act is Defined by The EPA as ``The Resource Conservation and Recovery Act (RCRA) gives EPA the authority to control hazardous waste from the ``cradle-to-grave.'' This includes the generation, transportation, treatment, storage and disposal of hazardous waste. To achieve this, EPA develops regulations, guidance, and policies that ensure the safe management and cleanup of solid and hazardous waste, and programs that encourage source reduction and beneficial reuse.''\footnote{https://www.epa.gov/laws-regulations/summary-resource-conservation-and-recovery-act 2017}\vspace{1.5em}
	
	\pagebreak
	
	\paragraph{}
	\large
	This brings me to our goal or the reason for our case:
	\section{Goal:}
	\begin{quotation}
		\lettrine{R}{esponsibly protect citizens and the environment.}
		The Government in the status quo is not responsibly protecting its citizens and the environment as we will see through our:
	\end{quotation}
	
	\normalsize
	
	\section{Facts}
	
	\subsection{Fact 1: Frac Ponds Are Common}
	\textit{\ul{Infectious specialist Disease, Judy Stone, Forbes, Feb 23, 2017}\footnote{Judy Stone is an Infectious Disease specialist, experienced in conducting clinical research and the author of Conducting Clinical Research, the essential guide to the topic. https://www.forbes.com/sites/judystone/2017/02/23/fracking-is-dangerous-to-your-health-heres-why/}}
	\paragraph{}
	``\ul{Cough, shortness of breath and wheezing are the most common complaints of residents living near fracked wells. Toxic gases like benzene are released from the rock by fracking. Similarly, a toxic waste brew of water and chemicals is often stored in open pits, releasing volatile organic compounds into the air.} These noxious chemicals and particulates are also released by the diesel-powered pumps used to inject the water. An epidemiological study of more than 400,000 patients of Pennsylvania’s Geisinger clinic, done with Johns Hopkins School of Public Health, found a significant association between fracking and increases in mild, moderate and severe cases of asthma (odds ratios 4.4 to 1.5).''\newline
	\textbf{MPX: As we see from this evidence, waste from fracking is often stored in open ponds leading to many harms as stated in this article and in many others.}
	
	\subsection{Fact 2: Number is Growing}
	\textit{\ul{Susan Phillips, State Impact Pennsylvania, Nov. 11, 2014}\footnote{Susan Phillips tells stories about the consequences of political decisions on people's every day lives. She has worked as a reporter for WHYY since 2004. Susan's coverage of the 2008 Presidential election resulted in a story on the front page of the New York Times. https://stateimpact.npr.org/pennsylvania/2014/11/11/pennsylvanias-frack-ponds-now-number-more-than-500/}}
	\paragraph{}
	``From 2010 to 2013 the median area of drilling impoundments more than tripled, and the average area (which also includes small fluid reserve pits located right on the well pad) more than doubled. As of 2013, the total impoundment surface area measures nearly four million square meters, scattered across the Commonwealth.''\newline
	\textbf{MPX: As we saw from this evidence, these open pits are not diminishing but actually growing causing devastating effects to health and environment. Just as a clarification this piece is just about Pennsylvania.  To put that number in perspective for every 29,000 square miles in Pennsylvania 1 is covered in this toxic waste.  This leads to our harms or the reasons you should vote for our plan:}

\end{document}